\documentclass[acmsmall, nonacm, screen]{acmart}

\usepackage[utf8]{inputenc}
\usepackage[T1]{fontenc}
\usepackage{listings}
\usepackage{caption}
\usepackage{subcaption}
\usepackage{float}

\AtBeginDocument{%
  \providecommand\BibTeX{{%
    \normalfont B\kern-0.5em{\scshape i\kern-0.25em b}\kern-0.8em\TeX}}}

\makeatletter
\let\@authorsaddresses\@empty
\makeatother

\acmBooktitle{}

\title{CIS 670 Project Proposal: Mini-DBMS in Rust}

\author{Harrison Goldstein}
\author{Lucas Silver}

\begin{document}

\maketitle

\section{Overview}

For our CIS 670 project, we plan to implement a very simple Database Management System (DBMS) in
Rust. Our main goal is to learn about Rust and its type system, paying special
attention to the way that affine types provide protection against memory corruption and data
races.

We think that a implementing DBMS is a perfect way to explore Rust: many internal database
operations are both memory-bound and readily parallelized, which means there is lots of surface
area for bugs that Rust is uniquely equipped to prevent. In addition, DBMS systems are extremely
extensible. There are endless features and optimizations that can be added to build on a minimum
implementation. If we find that our base implementation is too simple (or if we realize that we
haven't gotten to use particularly fun Rust features) we can choose to extend the project on the
fly.

At the end of the project, we hope to have a clear picture of exactly what Rust's type system does
well (and where it might fall short).

\section{Project Specifics}

Our minimum viable product will include
\begin{itemize}
  \item a parser for a simple subset of SQL;
  \item a basic query planner that interprets a SQL query in relational algebra;
  \item an interpreter for query plans that operates on CSV files; and
  \item a simple command-line interface.
\end{itemize}

We anticipate that our minimum viable product will be a bit {\em too} minimal. This is where
things get fun. We hope to implement some or all of the following extensions:
\begin{itemize}
  \item Parallel algorithms for implementing certain kinds of queries
  \item A streaming interface for files that are too large to hold in memory
  \item Networking capabilities for remote queries
  \item A binary table format that is more compact than CSV
  \item Query optimizations based on relational algebra
  \item Table indexes (e.g. B+ tree)
  \item Transactions
\end{itemize}
If you have a sense for which of these additions you'd like to see, please let us know.

\end{document}
